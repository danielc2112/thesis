\subsection{Anforderungen}\label{sec:requirements}

Da die Skalierbarkeit der Anwendung einen hohen Stellenwert hat, werden im vorliegenden Abschnitt neben funktionellen Anforderungen auch Anforderungen an die Leistungsfähigkeit beschrieben.

%... Hierzu sollen in der Arbeit die Möglichkeiten der Integration von Recommendation- Engines und Suchindexen untersucht werden. Neben der Vorstellung der existieren- den Recommendation-Ansätze und Algorithmen, soll dafür vor allem das Suchergebnis- Boosting und die Kombination verschiedener Algorithmen zu diesem Zweck ausgearbeitet werden.

\subsubsection{Anwendungsfälle}\label{sec:userstories}

Aus der angestrebten Kombination von Empfehlungsdienst und Suchindexen ergeben sich zahlreiche Anwendungsfälle. Für jeden der vorab definierten Fälle beschreibt der Abschnitt, welcher Mehrwert sich ergibt, wie dieser geprüft werden kann und welche Aufgaben die beiden Bestandteile haben. Die Anwendungsfälle werden zudem durch Beispiele ergänzt, welche an den Verkauf von Konzertkarten angelehnt sind.% \todo[color=orange]{Ggf. Beispielbasis in der gesamten Arbeit so definieren}

\paragraph{Personalisierte Suche} Wie in Abschnitt \ref{sec:personalresultstheorie} beschrieben, kann die Relevanzberechnung beim Sortieren einer Dokumentenliste auch entsprechend eines Nutzerprofils angepasst werden. In der zu einer Anfrage gefundenen Dokumentenliste sollen dadurch die für den Nutzer relevanten Dokumente bevorzugt werden. Ob die erzielten Ergebnisse tatsächlich höheren Nutzen bieten, kann zum Beispiel durch eine Steigerung des ``Successful Session'' Maßes (vgl. Abschnitt \ref{sec:successfulsession}) gemessen werden.

Die Aufgabe der Profilbildung soll durch die Mittel der Empfehlungsdienste (vgl. Abschnitt \ref{sec:filtermethods}) integriert werden. Der Suchindex verarbeitet die Anfrage des Nutzers, findet alle relevanten Elemente und sortiert diese entsprechend der Relevanz (vgl. Abschnitt \ref{sec:searchrelevance}).

Sucht ein Nutzer zum Beispiel nach ``Musicals Hamburg'', so wäre es die Aufgabe des Suchindexes, ``Disneys König der Löwen'' und ``Caveman'' als relevante Ergebnisse zu finden. Da der eigentliche Suchausdruck hier keinen Schluss auf die Präferenz des Nutzers zulässt, sollte die Sortierung der Ergebnisse durch den Empfehlungsdienst entsprechend angepasst sein.

%A) Personalisierte Suche: Wenn man nach allgemeinen Begriffen wie "Geschenk" "Wein" oder "Kleid" sucht, bekommt man bereits durch andere Nutzer gelernte passende Empfehlungen in der Suche höher angezeigt. (Nutzer schaut sich Rotweine an und bekommt bei Suche nach Wein passende Rotweine angezeigt. Im Vergleich dazu ein Nutzer der sich Weißweine angeschaut hat)

\paragraph{Kontextbasierte Empfehlungen} Ohne eine konkrete (Such-)Anfrage des Nutzers können auf Basis des vorliegenden Nutzerprofils Empfehlungen generieriert werden.  Die Kombination mit einem Suchindex ermöglicht es zudem, inhaltliche Aspekte einfließen zu lassen. Bei der Empfehlung von Veranstaltungen wäre es dadurch zum Beispiel möglich, nur Veranstaltungen zu empfehlen, die noch nicht ausgebucht sind oder die in einem bestimmten Zeitraum liegen.

Der Empfehlungsdienst ist in diesem Szenario für die Zusammenstellung der relevanten Element zuständig, dem Suchindex fällt die Aufgabe der Anreicherung und Filterung dieser Elemente zu. Die Kontrolle der Ergebnisse kann zum Beispiel durch den Vergleich von Klickraten oder Conversion-Raten erfolgen.

\paragraph{Komplementär-Suche} Als Ergänzung zu einer bestehenden Liste von Elementen, zum Beispiel in einer Merkliste oder einem Warenkorb, ist es möglich, für den Nutzer relevante komplementäre Elemente zu finden. Wie bei den kontextbasierten Empfehlungen werden mögliche komplementäre Elemente vom Empfehlungsdienst gefunden und durch die Informationen des Suchindex ergänzt oder gefiltert.

Im Gegensatz zum vorangegangenen Anwendungsfall ist der Bezug zu der vom Nutzer getroffenen Auswahl maßgeblich. Hat sich der Nutzer zum Beispiel für ein Musical in Hamburg entschieden, scheint es sinnvoll ihm noch eine passende Hotelreservierung oder passende Restaurants zu empfehlen. Das Zusammenspiel von Suchindex und Empfehlungsdienst ist in diesem Zusammenhang vielversprechend. Der Suchindex kann den inhaltlichen Bezug sicherstellen und dem Beispiel folgend nur Restaurants in der richtigen Stadt und zum richtigen Zeitpunkt zu finden. Der Empfehlungsdienst kann auf dieser Basis dann entsprechend der Präferenzen des Nutzers die richtige Vorauswahl treffen. Die Messung der Conversion-Raten kann auch hier zur Kontrolle genutzt werden.

%B) Personalisiertes-Recommendation Widget:
% - In der rechten Spalte werden (passend zu den letzten besuchten Items oder durchgeführten Suchen) persönliche Empfehlungen gegeben
%2) Crosselling Widget im Warenkorb (Empfehlungen zu den Items im Warenkorb)

\paragraph{Cross-Selling} Nutzt man die in Abschnitt \ref{sec:neighborhoods} beschriebenen elementbasierten Nachbarschaftsmodelle innerhalb des Empfehlungsdienstes, so ist es möglich, für jedes Element weitere ähnliche Elemente zu finden. Die Ähnlichkeit der Elemente bezieht sich dann je nach Maß (vgl. Abschnitt \ref{sec:similarities}) vor allem darauf wie oft diese gemeinsam ähnlich bewertet wurden. Die daraus getroffenen Empfehlungen können für das sog. \textit{Cross-Selling} genutzt werden.

Sucht ein Nutzer nach ``Musicals Köln'' kann dadurch zum Beispiel die Empfehlung ausgesprochen werden, dass der Besuch des Kölner Doms oder eine Altstadtführung von vielen anderen Nutzern hinzu gebucht wurde. Im Gegensatz zu den komplementären Suchen, bei denen die Art der Komplementbildung durch entsprechende Vorkonfiguration eingeschränkt ist, bestimmen hier einzig die Resultate der Nachbarschaftsmodelle das Ergebnis.

Der Empfehlungsdienst erfüllt hierbei zwei Aufgaben. Er findet über die Nachbarschaftsmodelle relevante Kanidatenelemente und kann die daraus resultierende Liste in einem zweiten Schritt entsprechend des Nutzerprofils personalisiert sortieren bzw. filtern. Der Suchindex kann wie bei den kontextbasierten Empfehlungen zur Filterung und inhaltlichen Anreicherung genutzt werden.

%1) Alla "Nutzer die x gekauft haben haben auch y gekauft" oder simple "Find most simelar items to an item" Hier macht die Kombination von Suchindex und Recommendation auch Sinn weil:
%- Der Suchindex alle zur Ausgabe relevanten Daten eines Dokumentes hat
%- (Vorberechnung?)
%- Es einfach möglich ist die Items über die die Recommendation gemacht werden sollen über inhaltliche (regelbasierte) Suchqueries weiter einzuschränken oder zu boosten (e.g. Items die höheren Preis haben und in gleicher Kategorie sind höher boosten)

\subsubsection{Leistungsanforderungen}\label{sec:performancereq}

Der Bestimmung der Leistungsanforderungen wurde die mit Google Analytics gemessene\footnote{siehe: http://www.google.com/analytics/} Webseitennutzung vom 01. September 2012 bis 30. November 2012 zugrunde gelegt. In diesem Zeitraum wurden ca. 40 Millionen Seitenaufrufe von ca. 1,2 Millionen unterschiedlichen Nutzern verursacht. Die durchschnittliche Zahl von 500.000 Seitenaufrufen pro Tag war zudem nicht gleichbleibend konstant, sondern enthielt zwei Zugriffsspitzen mit bis zu 1 Million Zugriffen am Tag. Der Anteil der Seitensuche am Gesamtaufkommen beträgt ca. 10\%. Die Verteilung der Seitenaufrufe am Tag ist abhängig von der Tageszeit. Im Zeitraum von 8 Uhr bis 18 Uhr werden ca. 80\% aller Aufrufe getätigt. Im gesamten Zeitraum gab es ca. 13.000 Kaufabschlüsse bzw. Conversions. Der Anteil wiederkehrender Nutzer beträgt im Mittel 70\%, dies entspricht dem 15-fachen der in diesem Zeitraum registrierten Nutzer. %\todo[color=green]{GA Screen ergänzen}

\paragraph{Datenerhebung} Nimmt man an, dass zudem jede der besuchten Seiten für die Datenerhebung relevant ist und dass im Mittel zwei Events mit je 150 Byte pro Seitenaufruf aufgezeichnet werden (vgl. Abschnitt \ref{sec:tracker-impl}), ergibt sich pro Tag ein Datenaufkommen zwischen 150 MB und 300 MB. Wegen der zeitlichen Verengung muss die Datenerhebung bis zu 2000 Anfragen pro Minute bzw. ca. 35 Anfragen pro Sekunde in den Lastzeiten verarbeiten. Um sicherzustellen dass der Dienst auch unter sehr hoher Last stabil und schnell arbeitet, werden diese Anforderung um das 5-fache gesteigert. Dadurch ergibt sich die Anforderung, dass die Datenerhebung bis zu 10.000 Anfragen pro Minute verarbeiten können muss.

\paragraph{Personalisierung} Die im vorangegangenen Abschnitt beschriebenen Anwendungsfälle zur Personalisierung sind für ca. 50\% der Seiten relevant. \todo{möp} Beachtet man zudem die Zahl der wiederkehrenden Nutzer, so resultiert dies in ca. 125.000 Personalisierungsanfragen an durchschnittlichen Tagen und ca. 300.000 Anfragen an Tagen mit Zugriffsspitzen. Beachtet man auch hier die zeitliche Verengung der Anfragen, muss mit 500 Anfragen pro Minute bzw. 10 Anfragen pro Sekunde in den Lastzeiten gerechnet werden. Wie bei der Datenerhebung resultiert daraus die Anforderung, dass die Personalisierung bis zu 2.500 Anfragen pro Minute verarbeiten können muss. % \todo[color=green]{Lidl-Usecase ergänzen?}

%6500 Suchen pro Tag
%1Mio Seitenaufrufe pro Tag (Peak)
%Ø 500.000 Seitenaufrufe pro Tag
%50\% 
%Traffic / Nutzerzahlen - hochgerechnete Bandbreiten / Memorybedarf / Speicherbedarf
%Ø Käufe pro Nutzer , Ø Visits pro Monat , Anzahl angemeldeter Nutzer vs. Anzahl neuer Nutzer pro Monat
% Verteilung Traffic auf den Tag

