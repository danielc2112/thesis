\subsection{Schwierigkeiten von Recommendern}\label{sec:filterissues}

Alle Methoden des kollaborativen Filterns haben ihre Stärken und Schwächen, die wichtigsten Herausforderungen werden im folgenden Absatz zusammengefasst.

\paragraph{Cold-Start} Existieren von Nutzern keine oder nur wenige Bewertungen im System, können die im vorangegangenen Abschnitt beschriebenen Methoden den neuen Nutzer nur schwer zu vorhandenen in Relation setzen. Daraus folgt, dass keine oder nur sehr ungenaue Empfehlungen generiert werden können. Ähnliches gilt für Elemente für die nur wenige Bewertungen vorliegen. Dieses Problem des sog. \textit{Cold-Start} kann durch die Gewinnung von zusätzlichem impliziten Informationen oder durch die Ergänzung von domänenspezifischem Wissen verringert werden (vgl. \citep{claypool99}).  In dem von \citep{Steck:2010:TTR:1835804.1835895} entwickelten Ansatz wird zudem auch in fehlenden Bewertungen ein gewisser Informationsgehalt und damit das Potential zur Verbesserung der Empfehlungen gezeigt. Angewandt wird dies zum Beispiel in \citep{Toscher:2008:INA:1722149.1722153} durch die Kombination von Nachbarschafts- und Matrixfaktorisierungsmodellen.

\paragraph{Dünnbesetzte Matrizen} Überträgt man die zu 6.3\% gefüllte \textit{User-Item} des 100K MovieLense Datensatzes\footnote{http://www.grouplens.org/node/73, Enthält Bewertungen von 943 Nutzer für 1682 Filme} auf Anwendungsfälle mit einer Million Elementen, so wird schnell klar, dass Annahmen über 60.000 Bewertungen pro Nutzer eher unrealistisch sind. Abhängig von der Größe des Systems wird es daher immer schwerer, ähnliche Nutzer ausschließlich über gemeinsam bewertete Elemente zu finden. Das daraus resultierende Problem der \textit{Neighbor transitivity} bezeichnet den Fall in dem aufgrund der zu geringen Datenmenge Nutzer mit ähnlichen Interessen nicht gefunden werden können, weil für sie keine sich überschneidenden Bewertungen vorliegen.

Die in Abschnitt \ref{sec:svd} vorgestellten Methoden der Matrixfaktorisierung bieten eine mögliche Lösung. Da man über die trainierten Modelle jeden Nutzer mit jedem anderen Nutzer und jedem Element in Relation setzen kann, wirken sich die fehlenden Daten nur noch auf die Genauigkeit des Modells aus, nicht aber auf dessen Fähigkeit überhaupt Empfehlungen zu generieren. Ein weiterer Ansatz ist die Kombination mit anderen nicht-kollaborativen Techniken. \citep{Koren:2009:MFT:1608565.1608614,claypool99}

\paragraph{Graue Schafe} Ein weiteres Problem in kleinen und mittleren Systemen ist nach \citep{claypool99} das der ``grauen Schafe''. In diese Gruppe fallen demnach alle Nutzer die in keine Gruppe fallen und für die es folglich nicht möglich ist, ähnliche Nutzer zu finden. Folglich können diese Nutzer keinen oder nur sehr geringen Nutzen aus den Empfehlungen ziehen. In \citep{claypool99} wird ein hybrides System auf inhaltsbasierten und kollaborativen Modellen zur Verbesserung der Empfehlungen in diesen Fällen erfolgreich evaluiert. \citep{Burke:2002:HRS:586321.586352}

\paragraph{More-of-the-same} Existieren von einem Element verschiedene Variationen in der \textit{User-Item} Matrix, ist es dem System nicht möglich, diese indirekte Verbindung zu erkennen. Aus der Präferenz für die E-Book-Ausgabe eines Buches kann folglich nicht die Ähnlichkeit zu anderen Nutzern gefunden werden, die die gedruckte Form des Buches bewertet haben. Gleichzeitig sind sich die Variationen oft trotzdem zu ähnlich, so dass die Empfehlungen aus offensichtlichen Elementen zusammengesetzt sind und für den Nutzer wenig Überraschungen bzw. sehr geringen Nutzen bieten. Zur Verbesserung konnte zum Beispiel zusätzliche inhaltliche Diversifikation der generierten Empfehlungen genutzt werden. \citep[Kap. 3]{rs}

\paragraph{Rich-gets-richer}\label{sec:richgetsricher} Dem Ziel, mit Hilfe von Recommender-Systemen die Diversität der vom Nutzer wahrgenommenen Elemente zu vergrößern, steht der sog. \textit{Short-Head vs. Long-Tail} oder \textit{Rich-gets-richer} Effekt gegenüber. Dieser tritt auf, wenn Bewertungen nicht gleichmäßig auf die Elemente verteilt sind. So beziehen sich zum Beispiel im MovieLense Datensatz\footnote{http://www.grouplens.org/node/73} 33\% der Bewertungen auf weniger als 10\% der Filme. Erschwert wird das Problem durch seine geringe Sichtbarkeit in den populären Metriken zum Vergleich von Recommender-Modellen. Die von \citep{Cremonesi:2010:PRA:1864708.1864721} durchgeführte Evaluation zeigt, dass alle Methoden anfällig für diese Effekte sind und das der Ausschluss der populärsten 2\% der Elemente vom Training zur Verbesserung der Empfehlungen der restlichen 98\% führen konnte. In \citep{Yin:2012:CLT:2311906.2311916} wird eine graphenbasierte Methode zur Verbesserung der Diversität in den Empfehlungen vorgeschlagen.

\paragraph{Manipulation} Die Fähigkeit eines Recommender-Systems das Verhalten der Nutzer zu adaptieren stellt zugleich eine große Angriffsfläche für gezielte Manipulationen dar. Diese haben entweder die gezielte Verstärkung (\textit{product push}) oder Verminderung (\textit{product nuke}) der Häufigkeit mit der ein Element empfohlen wird zum Ziel. Die Bandbreite der möglichen Attacken hängt dabei vom Grad der Kenntnis die der Angreifer über das System hat ab. Die verschiedenen Angriffsformen, deren mögliche Auswirkungen und Maßnahmen werden in \citep{hb_25} beschrieben. \todo[color=green]{Formen von Attacken oder Maßnahmen könnten ergänzt werden}

%Neben den wohlgekannten und durch entsprechende Metriken messbare Problemen ergeben sich auch zahlreiche praktische Probleme.
%<Nutzer wollen keine expliziten Ratings geben>
% Fluch der Dimensionalität
% 100 Datenpunkte füllen 2 dimensionalen Raum gut aus, um  eine ähnliche Abdeckung in einem 10 dim. Raum zu erreichen braucht man 10^20 Punkte
%%%%%%%%%%%%%%%
%- Allgemeine Recommendation Challenges und Ansätze (Sparse, Grey Sheep, "rich-gets-richer"...)
%- Latenzzeit für das lernen (Nutzer hat grad was angeschaut)
%- Lösung wie dus gemacht hast (Real time learning)
%- Item based ohne Nutzerid
%- Zeitliche Aspekte (Weihnachtsgeschäft etc) - (hattest du ja ausgeführt)
%"Collaborative recommenders work best for a user who fits into a niche with many neighbors of similar taste. The technique does not work well for so-called “gray sheep” (Claypool, et al. 1999), who fall on a border between existing cliques of users. "  
%Ahn07 - ColdStart Problem "beheben" \\
%Jin12 - LongTail \\
%Cacheda11 - ``Problemüberblick''