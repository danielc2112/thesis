
\newglossaryentry{AJAX}
{
  name=AJAX,
  description={ist ein Akronym for ``Asynchronous JavaScript and XML''. Es wird zur Datenübertragung zwischen Browser und Server genutzt. Die Übertragung kann auch sowohl synchron als auch asynchron erfolgen und die ausgetauschten Daten sind in der Regel per XML oder JSON serialisiert}
}
\newglossaryentry{REST}
{
  name=REST,
  description={ist ein Akronym für ``Representational State Transfer''. Es beschreibt die Nutzung von HTTP-basierten zustandslosen Diensten}
}
\newglossaryentry{ID}
{
  name=ID,
  description={bezeichnet eine Nummer die genutzt wird um Entitäten eindeutige zu referenzieren}
}
\newglossaryentry{SKU-ID}
{
  name=SKU-ID,
  description={ist ein Akronym für ``Stock keeping unit''. Es beschreibt eine ID, die die eindeutige Zuordnung zu einer bestimmten Bestandseinheit (Stückgut) und somit auch ihre Wiedererkennung ermöglicht}
}
\newglossaryentry{JSON}
{
  name=JSON,
  description={ist ein Akronym für ``JavaScript Object Notation''. Es ist ein für Menschen und Maschinen lesbares Dateiformat zum Datenaustausch und soll aus gültigen JavaScript bestehen}
}
\newglossaryentry{Recommender}
{
  name=Recommender,
  description={sind automatisierte Empfehlungstechnologien. Recommender filtern Informationen auf der Basis statistischer Daten, um damit Elemente (Produkte) aus einer Menge von Alternativen zu bestimmen, die entweder zum Nutzer oder zu einem anderen Element passen (vgl. Abschnitt \ref{sec:collaborativefiltering})}
}

\newglossaryentry{Servlet}
{
  name=Servlet,
  description={ beschreibt eine Java - Technologie um dynamische Webinhalte zu erstellen. Servlets laufen auf einem Webserver, um auf HTTP Anfragen dynamische Antworten zu erstellen}
}
\newglossaryentry{Tracking}
{
  name=Tracking,
  description={beschreibt das Erstellen eines Protokolls über das Nutzerverhalten auf einer Webseite}
}
\newglossaryentry{Webservice}
{
  name=Webservice,
  description={bezeichnet eine Technik zur Anwendungskopplung in heterogenen Systemen über standardisierte Protokolle. Der Datenaustausch erfolgt in der Regel über HTTP und die ausgetauschten Daten werden üblicherweise mittels XML oder JSON serialisiert}
}
\newglossaryentry{Reverse Proxy}
{
  name=Reverse Proxy,
  description={ist ein Netzwerkdienst über den der externe Zugriff auf Netz-interne Dienste gesteuert wird. Reverse Proxies werden u.a aus Sicherheitsgründen und zur Optimierung der Netzwerkleistung genutzt.}
 } 