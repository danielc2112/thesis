~
\vspace{5cm}
% siehe http://en.wikibooks.org/wiki/LaTeX/Document_Structure#Abstract

\begin{abstract}
%\begin{large}
%\textbf{Kurzfassung} \\ \\
%\end{large}

Die Integration von Suchtechnologien mit den Methoden des maschinellen Lernens bietet verschiedene Möglichkeiten um Suchergebnisse umfangreich zu personalisieren und so die Qualität der Suche für den Nutzer zu steigern. In dieser Arbeit werden die Grundlagen beider Technologien vorgestellt und zwei Möglichkeiten zur Integration, unter Berücksichtigung möglicher Herausforderungen bei der Skalierung, untersucht. Gegenübergestellt werden dabei elementbasierte Ähnlichkeitsmaße die mittels Webservice die Personalisierung einer Suche ermöglichen und faktorenbasierte Modelle welche die Personalisierungsberechnung direkt in der Suche integrieren. Verglichen werden die Leistungswerte und das Skalierungsverhalten, sowie die erzielbaren Qualität beider Lösungen. Die vorgestellte faktorenbasierte Personalisierung erwies sich dabei als qualitativ gleichwertige Alternative mit verbesserten Skalierungseigenschaften.

\end{abstract}
\newpage
