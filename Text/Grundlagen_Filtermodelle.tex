\subsection{Filtermodelle}
\label{sec:filtermethods}

Um die in Abschnitt \ref{sec:cf_overview} beschriebenen kollaborativen Filtermethoden zu nutzen, stellt sich das Problem, wie die Ähnlichkeit von Nutzern oder Elementen bestimmen werden kann und wie Empfehlungen für einen Nutzer erzeugt werden. Die dafür nötigen Modelle sollen in den folgenden Abschnitten näher erläutert werden.

Grundlage der im Folgenden beschriebenen Methoden ist eine \textit{User-Item} Matrix $R$, welche die Bewertung aller Nutzer $U$ für die Elemente (Produkte) $P$ enthält. Die Wahl des Wertebereichs hängt dabei von der Applikation ab. Ein Beispiel für eine solche Matrix wird in Tabelle \ref{tab:user-item-ratings} gezeigt.
\begin{table}
  \centering
  \begin{tabular}{ | l || c | c | c | c | c | c | c | }
    \hline
           & \sturz{Item1 } & \sturz{Item2}  & \sturz{Item3}  & \sturz{Item4}  & \sturz{Item5}  & \sturz{Item6}  & \sturz{Item7}  \\ \hline
User1 &    5.0 & 3.0      & 2.5     &   ?        & & & \\				
User2 &    2.0 & 2.5      & 5.0     &  2.0    & & & \\
User3 & 2.5	& & &	4.0 &	 4.5	& &	5.0 \\
User4 & 5.0	& &	3.0	& 4.5 & &	4.0 &	 \\
User5 & 4.0	&3.0 &	2.0 &	4.0 &  3.5 & 4.0	& \\
    \hline
  \end{tabular}
  \caption{\footnotesize Beispiel-Matrix für User-Item Ratings (vgl. \citep[Tabelle 2.1, S. 14]{rs})}
  \label{tab:user-item-ratings}
\end{table}

\subsubsection{Ähnlichkeitsmaße}
\label{sec:similarities}

\paragraph{\addtoindex{Euklidische Distanz}} Die naheliegendste Form zur Bestimmung der Ähnlichkeit zwischen zwei Spalten oder zwei Zeilen der User-Item Matrix ist es, deren Abstand im $n$-dimensionalen euklidischen Raum, gem. Formel (\ref{form:eukildsim}) zu nutzen.
\begin{align}
\label{form:eukildsim}
dist(a,b) & = \sqrt{\sum_{i=1}^{n} (a_i - b_i)^2} \\
sim(a,b) & = \frac{1}{1+dist(a,b)} \label{form:disttosim}
\end{align}

Hierbei ist $n$ die Anzahl der Dimensionen und $a_i$ bzw. $b_i$ beziehen sich auf das  $i$-te Attribut der Objekte, resp. die Ratings der Nutzer. Um den Distanzwert zu einem Maß der Ähnlichkeit mit einem Wertebereich von $1$ (starke Korrelation) bis $0$ (keine Korrelation) umzuformen, kann Formel (\ref{form:disttosim}) genutzt werden.

Aus der Verallgemeinerung dieser Berechnung, der sog. \textit{Lr-Norm} bzw. dem \textit{Minkowski Abstand}, ergeben sich weitere Abstandsmaße. Die sog. \textit{L1-Norm} (auch \textit{City-Block-} oder \textit{Manhattan-Distanz}) entspricht $r=1$, $r=2$ entspricht dem o.g. euklidische Abstand und $ r=\infty $ entspricht dem \textit{Tschebyscheff-Abstand}. \citep{hb_02}
\begin{align}
\label{form:minkowskisim}
dist(a,b) & = \sum_{i=1}^{n} (\left| a_i - b_i \right|^r)^\frac{1}{r}
\end{align}
% http://www.fernuni-hagen.de/imperia/md/content/ls_statistik/kurse/00883_lp2.pdf
% Anwendung finden die verschiedenen Abstandsmaße zum Beispiel in XXXXXX \todo{Anwendungsbeispiele raussuchen}

\paragraph{\addtoindex{Pearson-Korrelation}} Ein Problem bei der Berechnung mit der euklidischen Distanz ist, dass die Mittelwerte und Varianzen der Bewertungen einzelner Nutzer voneinander abweichen können, obwohl diese vergleichbare Interessen haben (vgl. \citep[Kap. 2]{pci}). Dieser Mangel wird mit Hilfe der \textit{Pearson-Korrelation} (\ref{form:pearsonsim}) beseitigt.  Ihr Wertebereich reicht von $1$ (starke Korrelation) bis $-1$ (starke negative Korrelation). Vor Allem bei der Bestimmung von nutzerbasierten Ähnlichkeiten konnten mit ihr in vielen Fällen sehr gute Ergebnisse erzielt werden. Zudem existieren zahlreiche Erweiterungen, um zum Beispiel die Gewichtung von Übereinstimmungen bei der Bewertung von kontroversen Elementen stärker hervorzuheben. \citep[Kap. 2.1]{rs} \citep{hb_02}

\begin{align}
\label{form:pearsonsim}
sim(a,b) & = \frac{\sum_{p \in P} (r_{a,p}-\bar{r_a})(r_{b,p}-\bar{r_b})}{\sqrt{\sum_{p \in P} (r_{a,p}-\bar{r_a})^2 }\sqrt{\sum_{p \in P} (r_{b,p}-\bar{r_b})^2 }}
\end{align}

\paragraph{\addtoindex{Kosinus-Ähnlichkeit}}\label{sec:cossim} Ein weiterer Ansatz, der sich zum Standardmaß bei der Abbildung von Element- bzw. Item-Ähnlichkeit entwickelt hat, ist die \textit{Kosinus-Ähnlichkeit} (\ref{form:cossim}). Die Distanz zwischen zwei Vektoren entspricht dabei dem zwischen ihnen aufgespannten Winkel, entsprechend steigt die Ähnlichkeit von Vektoren, wenn diese in die gleiche Richtung zeigen. 
\begin{align}
\label{form:cossim}
sim(a,b) & = \frac{a \cdot b}{\|a\| \|b\|}
\end{align}
Der Wertebereich des erzeugten Ähnlichkeitsmaßes liegt zwischen $1$ (starke Korrelation) und $0$ (keine Korrelation) wenn die genutzten Ausgangsvektoren nur positive Werte haben. Dies ist zum Beispiel der Fall bei den oft üblichen 5 Sterne Rating-Skalen oder beim Vergleich von Textdokumenten anhand der Vorkommen einzelner Wörter. Das Maß reicht bis $-1$ für starke negative Korrelationen, wenn auch negative Werte genutzt werden. \citep{rs}[Kap. 2.2]

\paragraph{\addtoindex{Jaccard-Koeffizient}} Liegen Ratings nur als binäre Werte vor, kann die Ähnlichkeit zweier Elemente durch das Verhältnis der Schnittmenge zur Vereinigungsmenge dieser definiert werden. Der Wertebereich des sog. \textit{Jaccard-Koeffizienten} (\ref{form:jaccardsim}) liegt ebenso zwischen $1$ und $0$. Verwendung findet er auch, wenn die Werte wenig Informationen tragen und die Information, ob ein Nutzer eine Bewertung abgegeben hat, im Zentrum der Betrachtung steht oder durch die Rating-Werte Beziehungen zwischen Nutzern und Elementen (im Sinne eines Graphen) ausgedrückt werden. Erweitert wird der Jaccard-Koeffizent vom \textit{Tanimoto-} und vom \textit{\addtoindex{Dice-Koeffizienten}} (vgl. \citep{bogers09}). \citep[Kap. 3.1]{rs} \citep{pci}
\begin{align}
\label{form:jaccardsim}
sim(A,B) & = \frac{|A \cap B|}{|A \cup B|}
\end{align}

Welches der Distanzmaße für eine konkrete Anwendung genutzt werden sollte, kann nicht pauschal beantwortet werden. Durch empirische Analysen konnte allerdings gezeigt werden, dass bei der Bestimmung von nutzerbasierten Ähnlichkeiten die Pearson-Korrelation andere Metriken übertrifft. Beim Vergleich von Elementen wird sie von der Kosinus Ähnlichkeit übertroffen. In jedem Fall muss die Wahl eines Maßes immer mit einer entsprechenden Evaluation gegenüber anderen Maßen kontrolliert werden (vgl. Abschnitt \ref{sec:measures} u. \ref{sec:evaluation}) \citep[Kap. 2.1.2]{rs} \citep{Cacheda2011}.
% \paragraph{\addtoindex{Likelihood-Funktion}}
\subsubsection{Nachbarschaftsmodelle}\label{sec:neighborhoods}

\paragraph{Nutzer-basierte Modelle} Geht man nun davon aus, dass ähnliche Nutzer auch in der Zukunft eine ähnliche Meinung zu einem Element haben werden, kann man für einen Nutzer $u$ aus den vorliegenden Bewertungen ähnlicher Nutzer $\mathcal{N}_i(U)$ eine Bewertung für ein Element $i$ voraussagen ($pred(u,i)$). Die dabei in Betracht gezogenen anderen Nutzer werden auch als ``Nachbarschaft'' des Nutzers bezeichnet. Da diese zudem i.d.R. auf eine bestimmte Größe $k$ oder einen bestimmten Ähnlichkeits-Schwellwert limitiert wird, wird die Methode als \textit{k--nearest-neighbors} (k-NN) bezeichnet.

Um Empfehlungen für einen Nutzer aus den in Tabelle \ref{tab:user-item-ratings} gezeigten Ausgangsdaten abzuleiten, wird mit Hilfe der schon vorliegenden Ratings zunächst die Ähnlichkeit dieses Nutzers zu anderen berechnet (siehe Tabelle \ref{tab:user-user-sim}). Um $pred(u,i)$ aus diesen abzuleiten, werden die Ratings anderer Nutzer für dieses Element $r_{v,i}$ entsprechend der Ähnlichkeit zwischen den Nutzern aufsummiert und normiert:
\begin{align}
pred(u,i) & = \frac{ \sum_{v \in \mathcal{N}_i(U)} sim(u, v)*r_{v,i}}{ \sum_{v \in \mathcal{N}_i(U)} sim(u,v) } \label{form:calcpred}
\end{align}
Wie \citep{Herlocker:2002:EAD:593967.594047} zeigen, muss zudem ein weiterer Unterschied zwischen einzelnen Nutzern in Betracht gezogen werden. Auch wenn Nutzer generell ähnliche Interessen bzw. Meinungen haben, so kann es durchaus sein, dass Mittelwert und Varianz der Ratings dieser Nutzer sehr verschieden sind. Diese Gewichtung am Rating-Mittelwert $\overline{r_u}$ und der Varianz  $\sigma_u$ der Nutzer, dem sog. \textit{Z-score}, wird aus diesem Grund in den erweiterten Formeln (\ref{form:calcmeanpred}) und (\ref{form:calcmeanvarpred}) beachtet. \citep{hb_04,Huete:2012:UPA:2206442.2206675} 
\begin{table}
  \centering
  \begin{tabular}{ | l || c | c | c | c | c | c | c | }
    \hline
           & User2 & User3 & User4 & User5 \\ \hline
User1 &    0.203 &	0.286 &	0.667 & 0.472 \\	
    \hline
  \end{tabular}
  \caption{\footnotesize Aus Tabelle \ref{tab:user-item-ratings} mit der euklidischen Distanz abgeleiteter Ähnlichkeitsvektor für User1}
  \label{tab:user-user-sim}
\end{table}
\begin{align}
pred(u, i) & = \overline{r_u} + \frac{ \sum_{v \in \mathcal{N}_i(U)} sim(u, v)*(r_{v,i}-\overline{r_v}) } { \sum_{v \in \mathcal{N}_i(U)} sim(u,v) } \label{form:calcmeanpred} \\
pred(u, i) & = \overline{r_u} + \sigma_u \frac{ \sum_{v \in \mathcal{N}_i(U)} sim(u, v)*\frac{r_{v,i}-\overline{r_v}}{\sigma_v} } { \sum_{v \in \mathcal{N}_i(U)} sim(u,v) } \label{form:calcmeanvarpred}
\end{align}

\paragraph{Element-basierte Modelle} Neben der Bestimmung von ähnlichen Nutzern, kann mit den in der \textit{User-Item} Matrix vorliegenden Daten auch die Ähnlichkeit von Elementen bestimmt werden. Die Ähnlichkeiten bzw. Nachbarschaften der Elemente $\mathcal{N}_u(I)$ zu anderen kann dann, analog zu Formel (\ref{form:calcpred}), wie folgt zur Voraussage der Bewertungen genutzt werden: 
\begin{align}
pred(u,i) & = \frac{ \sum_{j \in \mathcal{N}_u(I)} sim(i,j)*r_{u,j}}{ \sum_{j \in \mathcal{N}_u(I)\mathcal{N}_i(U)} sim(i,j) } \label{form:calcpreditem}
\end{align}
\begin{table}
  \centering
  \begin{tabular}{ | l || c | c | c | c | c | c | c | }
    \hline
           & Item1 & Item2 & Item3 & Item5 & Item6 & Item7 \\ \hline
Item4 &    0.387 &	0.472 &	0.204 & 0.586 & 0.667 & 0.500 \\	
    \hline
  \end{tabular}
  \caption{\footnotesize Aus Tabelle \ref{tab:user-item-ratings} mit der euklidischen Distanz abgeleiteter Ähnlichkeitsvektor für Item4}
  \label{tab:item-item-sim}
\end{table}
Wie bei den nutzerbasierten Modellen sollte auch hier der Einfluss verschiedener Bewertungs-Mittelwerte und Varianzen ausgeglichen werden. Dies geschieht analog zu Formel (\ref{form:calcmeanpred}) und (\ref{form:calcmeanvarpred}). \todo{Quelle das ``Normalisierung'' auch bei item-item relevant ist} \todo{Ergebnisse der Rechnung einfügen}

Für die Abwägung zwischen nutzer- und elementbasierten Methoden gibt \citep{hb_04} die folgenden Kriterien an:

\begin{itemize}
\item \textit{Genauigkeit}: Abhängig von der Menge der Nutzer und Elemente im System schwankt die Zuverlässigkeit der Nachbarschaften. Ist die Anzahl der Nutzer im System größer als die der Elemente, so kann man davon ausgehen, dass elementbasierte Methoden kleinere aber zuverlässigere Nachbarschaften für die einzelnen Elemente produzieren und damit bessere Ergebnisse liefern und umgekehrt (vgl. auch \citep{Huete:2012:UPA:2206442.2206675} u. \citep{Herlocker:2002:EAD:593967.594047}) 
\item \textit{Effizienz} - Das Verhältnis von Nutzern und Elementen beeinflusst auch den Umfang der notwendigen Berechnung bei der Bestimmung von Nachbarschaften. \citep{linden03} zeigt zum Beispiel, dass die Grenzen der Skalierbarkeit schnell erreicht werden, wenn die Zahl der Nutzer die der Elemente stark überschreitet.
\item \textit{Stabilität} - Die Änderungshäufigkeit in der Menge der bewerteten Elemente bzw. die Fluktuation der Nutzer beeinflusst wie stabil Ähnlichkeiten sind. Ist die gewählte Basis ausreichend stabil können Ähnlichkeiten vorberechnet werden. 
\item \textit{Erklärbarkeit} - Die der Berechnung von elementebasierten Ähnlichkeiten zugrunde liegenden Elemente lassen sich leicht zur Erklärung der Ergebnisse nutzen und ermöglichen ggf. eine darauf basierende Interaktion mit dem Nutzer. Bei nutzerbasierten Modellen ist dies i.d.R. auch aus Gründen des Datenschutzes erheblich schwerer.
\item \textit{Zufälligkeit} - Die im ersten Punkt beschriebene Genauigkeit führt i.d.R dazu, dass bei elementbasierten Modellen oft weniger überraschende Vorschläge erzeugt werden. Bezieht man bei nutzerbasierten Modelle nur wenige Nachbarn zur Erzeugung der Vorschläge ein, ist die Wahrscheinlichkeit einer ``Überraschung'' höher. (vgl. Abschnitt \ref{sec:richgetsricher})
\end{itemize}

\paragraph{Nachbarschaftsgrößen} Unabhängig von der Methodenwahl muss die Größe der betrachteten Nachbarschaft $k$ begrenzt werden. Wählt man feste Werte, so sind Größen zwischen 20 und 50 (vgl. \citep{Herlocker:2002:EAD:593967.594047}) üblich. Kleinere Nachbarschaften werden wegen ihrer Anfälligkeit für Ausreißer nicht empfohlen, bei Größeren wiederum steigt i.d.R. die Fehlerquote. 

Neben festen Werten wird oft alternativ die Wahl eines Ähnlichkeits-Schwellwertes vorgeschlagen. Dabei werden alle Nachbarn einbezogen, deren Ähnlichkeit einen Maximalabstand nicht überschreitet. Da die Wahl des Schwellwertes nicht nur Auswirkungen auf den Umfang der Nachbarschaft, sondern auch auf die Abdeckung der berechenbaren Bewertungsvorhersagen hat, wird i.d.R (vgl. \citep{Herlocker:2002:EAD:593967.594047, Herlocker:1999:AFP:312624.312682}) von der alleinigen Verwendung abgeraten.

Die Wahl der konkreten Größe ist bei beiden Methoden zudem in jedem Fall ein Kompromiss zwischen Genauigkeit, Zufälligkeit und Effizienz. 

\paragraph{Vorteile von Nachbarschaftsmodellen} Dass zum Vergleich mit anderen Methoden, keine langwierige Trainingsphase notwendig ist, ist  ebenso wie die leichte Nachvollziehbarkeit der  Methodik ein Vorteil von Nachbarschaftsmodellen. Die Möglichkeit, Empfehlungen durch eine Erklärung zu ergänzen, ist ein weiterer Vorteil dem bei der Bildung des Nutzervertrauens besonders viel Gewicht zukommt (vgl. \citep{hb_15}). Die Stabilität der zugrunde liegenden Ähnlichkeitsmatrizen und der Effizienzgewinn durch die mögliche Vorberechnung der Nachbarschaften sind zudem bei großen Systemen von Vorteil (vgl. \citep{linden03}).\citep{hb_04} \todo[color=green!40]{Weiterführend: Regression vs. Classification aus HB04 evt. ergänzen}
%Sarwar01  -> Item-base zu bevorzugen \\
%\citep{bogers09} -> User-Based zu bevorzugen \\
%Prediction-accuracy with in \citep{Huete:2012:UPA:2206442.2206675}
%\todo[color=green!40]{ggf. SlopeOne ergänzen}

%\subsubsection{SlopeOne}
%
%\citep[S 41]{rs}
%http://lemire.me/fr/abstracts/SDM2005.html
\subsubsection{Matrixfaktorisierung}
\label{sec:svd}

\paragraph{Merkmal-basierte Empfehlungen} Bei den Methoden der Matrixfaktorierung geht man davon aus, dass zusätzliche Merkmale der Nutzer und Elemente (\textit{Features}) aus den Einträge der \textit{User-Item} Matrix abgeleitet werden können. Die Bandbreite erstreckt sich von sehr anschaulichen Merkmalen, wie etwa dem Genre bei Büchern, Filmen oder Musik, über schwer definierbare Merkmale wie etwa Qualität bis hin zu uninterpretierbaren. Der Erfolg dieses Ansatzes wurde zum Beispiel beim 2006 ausgeschriebenen Netflix-Preis, zur Generierung von Filmempfehlungen, unter Beweis gestellt (vgl. \citep{Koren:2009:MFT:1608565.1608614}).
%\todo[color=green]{PureSVD erläutern?}

Um die in den Einträgen der Matrix verborgenen Merkmale zu berechnen, werden diese in einem $f$-dimensionalen Raum als Produkt von Nutzer- und Elementvektoren abgebildet. Die Dimension des Raumes entspricht der Anzahl der Merkmale, beim o.g. Beispiel lag diese zwischen 100 und 500. Die Einträge des Elementvektors $q_i \in \mathbb{R}^f$ drücken aus, zu welchem Grad --- positiv oder negativ --- dessen Eigenschaften dem Merkmal entsprechen. Der Nutzervektor $p_u \in \mathbb{R}^f$ korreliert entsprechend die Interessen des Nutzers mit diesen Merkmalen. Bezieht man zudem den Rating-Mittelwert $\mu$, sowie element- und nutzerspezifische Abweichungen $b_x$ ein, so ergibt die mögliche Bewertung eines Nutzer $u$ für ein Element $i$:
\begin{align}
pred(u,i) & = \mu + b_i + b_u + q_i^T p_u \label{form:calcpredsvd}
\end{align}

Die dafür benötigten Parameter $q$, $p$ und $b$ werden in einer Trainingsphase aus den vorhandenen Bewertungen gelernt. Dies geschieht mit den Methoden der \acf{SVD}  \citep{golub65} durch die Minimierung der mittleren quadratischen Abweichung zwischen existierenden und vorausberechneten Bewertungen:
\begin{align}
\min_{q*,p*,b*}{ \sum_{(u,i) \in \mathcal{K})} (r_{ui} -\mu-b_i - b_u - p_i^T q_u)^2  + \lambda ( \|q_u\|^2 + \|p_i\|^2 + b_u^2 + b_i^2) \label{form:trainsvd}   }
\end{align}

Die Menge $\mathcal{K}$ entspricht dabei allen vorliegenden Bewertungen $r_{ui}$. Die Konstante $\lambda$ wird genutzt um das sog. \textit{overfitting} zu verhindern. Sie stellt sicher, dass das abgeleitete Modell generisch bleibt.  \citep{Koren:2009:MFT:1608565.1608614, hb_05}

\paragraph{Trainingsmethoden} Zur Durchführung des Trainings kann eine der beiden im Folgenden beschriebenen Methoden verwendet werden.

Beim \textit{stochastischen Gradientenverfahren} \citep{funk2006} wird das Minimierungsproblem durch einen Gradientenabstieg gelöst. Die Richtung des Abstieges  ergibt sich mit Hilfe der vorliegenden Trainingsdaten aus der Abweichung $e_{ui}$ zwischen den tatsächlichen und vorausberechneten Werten (siehe Formal (\ref{form:graddec-1}) - (\ref{form:graddec-3})). Der Faktor $\gamma$ entspricht dabei der Lernrate bzw. Schrittweite. Entsprechend der Anzahl der zu ermittelnden Merkmale wird das Training $f$-mal, jeweils bis zur Konvergenz, durchgeführt. \citep{funk2006,Langford09,hb_05}
\begin{align}
e_{ui} & =  r_{ui} - pred(u,i) \label{form:graddec-1} \\
q'_i & \gets q_i + \gamma (e_{ui} p_u - \lambda q_i ) \label{form:graddec-2} \\
p'_u & \gets p_u + \gamma (e_{ui} q_i - \lambda p_u) \\
b'_i & \gets b_i + \gamma (e_{ui} - \lambda b_i) \\
b'_u & \gets b_u + \gamma (e_{ui} - \lambda b_u) \label{form:graddec-3}
\end{align}

Die Methode der \textit{Alternating Least Squares} \citep{Bell:2007:SCF:1441428.1442050} verfolgt einen anderen Ansatz zur Lösung des Minimierungsproblems. Da $ p_i^T q_u $ nicht konvex ist, kann das resultierende Gleichungssystem nicht vollständig gelöst werden. Wird $p_i$ oder $q_u$ als konstant angenommen, ist eine Approximation der nicht konstanten Werte durch die Methode der kleinsten Quadrate möglich. Im Trainingsverlauf werden deshalb abwechselnd die $p_i$ und $q_u$ konstant gehalten, um die jeweils anderen anzupassen. Wegen des aufwendigeren Ablaufs konvergiert das Verfahren langsamer als das zuvor beschriebene Gradientenverfahren, der Mangel wird bei großen Datenmengen aber durch eine bessere Parallelisierbarkeit ausgeglichen. \citep{Bell:2007:SCF:1441428.1442050, hb_05} % \todo[color=green]{Pseudocode aus Louppe10 ?} \todo[color=green]{Temporale Effekte dazu ?}

\paragraph{Vorteile der Matrixfaktorisierung} Ein wichtiger Vorteil der durch die Matrixfaktorisierung erzeugten Modelle ist die Approximation der in der \textit{User-Item} Matrix enthaltenen Informationen in erheblich kompakterer Form. Zudem ermöglicht sie, dass über die Anzahl der zu lernenden Merkmale die Größe des resultierenden Modelle getsteuert werden kann. Die Erweiterbarkeit der Ausgangsformel (\ref{form:calcpredsvd}) ist ein zusätzlicher Vorteil. So konnten mit Erweiterungen, die zum Beispiel temporale Effekte oder nicht-lineare Zusammenhänge in den Daten ausnutzen, in verschiedenen Anwendungsfällen zusätzlich verbesserte Ergebnisse erzielt werden. \citep{hb_05,Vozalis:2007:USD:1243505.1243639}

% SVD - Abbildung HB S. 46
%\citep{Koren:2009:MFT:1608565.1608614} \citep{Vozalis:2007:USD:1243505.1243639}
% Brand04 / __Cremonesi10 erweiterte Erklärung
%Cacheda11 - bestätigt SVD schlägt Alternative vor

%\subsubsection{Graphen Modelle}\newpage
%\todo[color=green!20]{Iterative Methoden ggf. ergänzen bzw. Informationen zu iterativen Updates der Modelle hinzufügen - siehe Entwurf->Problemstellung}
%\todo[color=green!40]{Weitere Ansätze ergänzen? zB. Graphen-Modelle? Bolzmann-Dings?}
% Boltzmann: Louppe10
