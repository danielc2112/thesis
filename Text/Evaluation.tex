\section{Evaluation}\label{sec:evaluation}

\textit{Wie wird gemessen, welche Ergebnisse waren zu erwarten, was wurde erreicht. Warum gibt es Abweichungen, welche Probleme enthält die Messmethode.\todo{raus}}

Im folgenden Abschnitt wird die implementierte Lösung hinsichtlich der Leistungs- und Qualitätsanforderungen (vgl. Abschnitt \ref{sec:requirements}) evaluiert. Da zum Zeitpunkt der Fertigstellung der Arbeit noch kein ausreichend umfangreicher realer Datensatz zur Verfügung stand, wurde der frei verfügbare MovieLens 1M \citep{movielens1m} Datensatz genutzt. Dieser besteht aus ca. 1 Million Nutzerbewertungen von 3.900 Nutzern für 6040 Kinofilme. Er wurde gewählt, da er bereits in zahlreichen Publikationen zur Ergebnisevaluation genutzt wird und so eine gute Vergleichbarkeit bei den Ergebnissen erzielt werden kann.

\subsection{Ergebnisse}

\subsubsection{Leistung}

\begin{itemize}
\item Performance Tracker (req/s) - (concurrent con)
\item Performance Solr Personalisierung 1 single core (req/s) - (concurrent con)
\item Performance Solr Personalisierung 1 multi core / single recommender (req/s) - (concurrent con)
\item Performance Solr Personalisierung 1 multi core / multi recommender (req/s) - (concurrent con)
\item Performance Solr Personalisierung 2 single core (req/s) - (concurrent con)
\item Performance Solr Personalisierung 2 multi core (req/s) - (concurrent con)
\end{itemize}

\subsubsection{Qualität}

\begin{itemize}
\item Qualität Personalisierung 1 - Metrik RMSE (Euclid, Pearson, Kosinus, Jaccard)
\item Qualität Personalisierung 1 - Nachbarschaftsgröße (20,30,40,50)
\item Qualität Personalisierung 2 - Metrik RMSE
\end{itemize}

\subsection{Diskussion}

Cremonesi10 - Abwägung RSME / Pression / Recall Messung bei Top-N Recommendern \\
Howe08 - Abwägung der versch. Distanzmaße zwischen Datensätzen

Skalierbarkeit evt. mit weiterem Datensatz ``testen'': \\
\url{http://aws.amazon.com/datasets/6468931156960467} -> (Subset: \url{http://labrosa.ee.columbia.edu/millionsong/tasteprofile}) \\

Signifikanz: \url{http://www.mitp.de/imperia/md/content/vmi/1634/1634_kapitel_20.pdf} \\
Joachim05 -> Wilcoxon test