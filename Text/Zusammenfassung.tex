
\section{Zusammenfassung}\label{sec:results}

%\textit{Abriss der Arbeit, was wurde erreicht bzw. gelernt. An welcher Stellen kann weitergearbeitet werden.}

\subsection{Fazit}

In dieser Arbeit wurden die Möglichkeiten zur Integration von Suchindexen und Empfehlungsdiensten untersucht. Unter Berücksichtigung der Skalierbarkeit wurden zu diesem Zweck zwei mögliche Lösungswege betrachtet. Diese ergänzen die bekannten Personalisierungsmethoden (vgl. Abschnitt \ref{sec:personalresultstheorie}) durch die direktere Integration der Methoden des kollaborativen Filterns mit Suchindexen. Zum Einen wurden die Ergebnisse eines Empfehlungsdienstes genutzt um Suchergebnisse zu personalisieren. Im zweiten Ansatz wurden Möglichkeiten zur direkten Empfehlungsbildung im Suchindex untersucht, um beide Dienste vollständig integriert nutzen zu können.

In der im Rahmen der Arbeit entwickelten Beispielapplikation wurden beide Personalisierungslösungen hinsichtlich ihrer Qualität und Leistungsfähigkeit evaluiert. Dabei hat sich gezeigt, dass die bekannten und oft untersuchten Algorithmen des kollaborativen Filterns (vgl. Abschnitt \ref{sec:neighborhoods}) bezüglich der Qualität bessere Ergebnisse liefern als die auf Matrixfaktorisierung basierende Personalisierungslösung. Durch die direktere Integration in Apache Solr liefern diese allerdings erheblich bessere Leistungswerte.

Umgesetzt wurde die Beispielapplikation auf Basis der quelloffenen Software Apache Solr und Apache Mahout.
\newpage
\subsection{Ausblick}

Obwohl im Rahmen dieser Arbeit zwei Möglichkeiten zur Personalisierung von Suchergebnissen erfolgreich implementiert wurden, verbleiben einige Fragen zur weiteren Betrachtung. Die in der Evaluation festgestellten Skalierungsprobleme von Apache Solr bei niedrigem Parallelisierungsgraden, mögliche Probleme der Zwischenspeichereffizienz bei der Personalisierung und die Optimierung der Parameter bei der Modellberechnung bedürfen weiteren Untersuchungen vor einem praktischen Einsatz. Bei der Verwendung der Personalisierung mittels Webservice gilt es, das Problem der disjunkte Kanidatenlisten genauer zu untersuchen.

Im Umgang mit dem Besucher bzw. Kunden einer Webseite sind im Zusammenhang mit den vorgestellten Personalisierungskonzepten Aspekte zu möglichen Einflüssen auf die Verkaufsdiversität (vgl. \citep{Fleder09}) und Möglichkeiten zur Integration von Kritikmechanismen (vgl. \citep{hb_13}). umbetrachtet geblieben.

Bei den eingesetzten Algorithmen ergeben sich ebenfalls Anknüpfungspunkte. Im Bereich der Matrixfaktoriersung existieren zahlreiche Erweiterungen zur Integration von implizitem und explizitem Feedback \citep{Joachims05} und zur Beachtung des temporalen Kontextes \citep{Boughareb11}. Daneben stellt die Kombination mehrerer Empfehlungsalgorithmen in einer Lösung, wie sie zum Beispiel bei der Netflix-Price-Competition notwendig war \citep{netflix2012_2}, einen weiteren wichtigen Aspekt zur Steigerung der Empfehlungsqualität dar. Wie \citep{Forbes11} zeigt, gilt dies ebenso  für die Kombination verschiedener Empfehlungskonzepte. 

Mit diesen zahlreichen Aspekten sollte die Integration eines Empfehlungssystems sicherlich als stetiger Prozess betrachtete werden, der durch neue Erkenntnisse und Algorithmen aus der Forschung vorangetrieben werden kann.



%\begin{itemize}
%\item Ggf. ``Lesezeit'' als Maß einbringen bzw. explizite und implizites Feedback (siehe Joachims05 - Abschnitt 2)
%\item Annahme das sich die Präferenzen der Nutzer über die Zeit nicht änder ist ggf. falsch <-> Temporaler Kontext aus Boughareb11 
%\item Auswirkungen auf die Verkaufsdiversität Fleder09
%\item Ausblick ``Rollout'' \citep{netflix2012_2}
%\item Ausblick ``Similarity Search in High Dimensions via Hashing'' (LSH) als mögliche Erweiterung
%\item Ausblick ``Content boosted matrix factorization'' with in Forbes11
%\item Ausblick andere Suchlösungen - Elastik Search etc..
%\end{itemize}
