\subsection{Bildung von Empfehlungen}\label{sec:myrecommend}

Da die verfügbaren Suchlösungen (vgl. Abschnitt \ref{sec:solr}) die Relevanzberechnung nicht ausschließlich \textit{TF-IDF}-basiert durchführen (vgl. Abschnitt \ref{sec:searchrelevance}) und auch andere Faktoren einbeziehen können, wird auch die direkte Integration der Berechnungen zur Personalisierung möglich. Die dafür notwendigen Anpassungen und Annahmen werden im Folgenden beschrieben und sollen bei der Umsetzung evaluiert werden. \todo[color=green]{untergliedern?}

Aufbauend auf den möglichen Maßnahmen zur Personalisierung der Relevanzberechnungen aus Abschnitt  \ref{sec:personalresultstheorie} können auch kollaborative Methoden einbezogen werden. Die \textit{TF-IDF}-basierte Relevanzberechnung aus Formel (\ref{form:docscore}) wird zu diesem Zweck wie in Formel (\ref{form:personalresultstheorie}) um  ein Nutzerprofil $u$ ergänzt. Entsprechend des Profils kann dann die Berechnung der Empfehlungen einbezogen werden (Formel (\ref{form:myscore0})). Der Faktor $\alpha$ wird zur Normalisierung des Empfehlungswertes genutzt.

Die Zwischenergebnisse der Matrixfaktorisierung bieten sich dafür besonders an, da sie im Gegensatz zu den auf Nutzer- oder Elementähnlichkeiten basierenden Methoden Element- und Nutzerfeatures in Form von Vektoren getrennt repräsentieren. Die Featurevektoren $p_u$ repräsentieren das Nutzerprofil, welches über ein Skalarprodukt  mit den Vektoren $q_i$ der einzelnen Dokumente direkt zur Relevanzberechnung genutzt werden kann. Formel (\ref{form:myscore1}) erweitert die ursprüngliche Relevanzberechnung aus Formel (\ref{form:docscore}) um die Empfehlungsberechnung aus Formel (\ref{form:calcpredsvd}). Zur Wahrung der Lesbarkeit beschreiben $v(d)$ und $v(u)$ die dem Dokument bzw. Nutzer zugehörigen Featurevektoren $q_i$ bzw $p_u$.
\begin{align}
\text{score}(q,d,u) & = \text{score}_{\text{tf-idf}}(q,d) * \alpha * \text{pred}(d, u)\label{form:myscore0} \\
\text{score}_{svd}(q,d,u) & = \text{score}_{\text{tf-idf}}(q,d) * \alpha * (\mu + b_d + b_u + v(d)^Tv(u)) \label{form:myscore1}
\end{align}

Das mit Hilfe der Matrixfaktorisierung gefundene Modell wird dadurch auf verschiedene Bestandteile des Systems verteilt. Durch die Erweiterung der im Suchindex gehaltenen Dokumente und die Ergänzung der Suchanfrage um die jeweiligen Featurevektoren wird es möglich, auch ohne einen zusätzlichen Empfehlungsdienst die Ergebnislisten zu personalisieren. Neben möglichen Leistungsverbesserungen sollen damit die Probleme disjunkter Kandidatenlisten (vgl. Abschnitt \ref{sec:disjunctcanidates}) in getrennten Systemen vermieden werden.

Um zudem die in Abschnitt \ref{sec:filterissues} beschriebenen \textit{Cold-Start} Probleme zu vermeiden und neu gewonnene Daten direkt einbeziehen zu können, wird die Berechnung wie in \citep{Paterek07} und \citep{Koren:2009:MFT:1608565.1608614} um weitere Informationsquellen ergänzt. Das Nutzerprofil kann dadurch um weitere Bewertungen $J_u$ ergänzt werden, ohne dass diese vollständig in das Modell einbezogen wurden:
\begin{align}
v(u) & \approx p_u  +(|J_u|+1)^{-0.5}*\sum_{j \in J_u} w_{j} \label{form:anonuser} \\
%\\
%pred(u,i) & = q_i^T p_u \\
%\\
% & \min_{q*,p*}{ \sum_{(u,i) \in \mathcal{K})} (r_{ui} - p_i^T q_u)^2  + \lambda ( \|q_u\|^2 + \|p_i\|^2 ) } \\
%\\
%pred(u,i) & = \mu + b_i + b_u + q_i^T p_u \\
%\\
%pred(u,i) & = \mu + b_i + b_u + q_i^T ( p_u +\frac{1}{\sqrt{|J_u|+1}}*\sum_{j \in J_u} w_{j} ) \\
%\\
%pred(u,i) & = \mu + b_i + b_u + q_i^T ( \frac{1}{\sqrt{|J_u|+1}}*\sum_{j \in J_u} w_{j} ) \\
%\\
%pred(u,i) & = \mu + b_i + b_u + q_i^T ( \frac{1}{\sqrt{|J_u|+1}}*\sum_{j \in J_u} q_{j} ) \\
%\\
%pred_u(1,4) & = \frac{2*0.203+4*0.286+4.5*0.667+4*0.472}{0.203+0.286+0.667+0.472} \\
%   & = 3.96 \\
%\\   
%pred_i(1,4) & = \frac{5*0.387+3*0.472+2.5*0.204}{0.387+0.472+0.204} \\
%    & = 3.63 \\
%\\
%pred_{svd}(1,4) & = q_4^T u_1 = 4.04
\end{align}

Wurde für einen Nutzer noch kein Nutzervektor $p_u$ berechnet, kann dieser ausgelassen werden, ohne die Möglichkeit zu verlieren, die Relevanzberechnung zu personalisieren.
\begin{align}
v(u) & \approx (|J_u|+1)^{-0.5}*\sum_{j \in J_u} w_{j} \label{form:anonuser}
\end{align}

Die hierfür genutzten Featurevektoren $w_j$ sind ebenfalls dokumentenspezifisch. Sie werden gemeinsam mit dem gesamten Modell mit den Methoden des \textit{stochastischen Gradientenverfahrens} oder den \textit{Alternating Least Squares} gelernt (vgl. Abschnitt \ref{sec:svd}).  \citep{Paterek07,Koren:2009:MFT:1608565.1608614}

Wie in \citep{Cacheda2011} gezeigt wird, ist die erhöhte Trainingsdauer ein Nachteil der Methode. Dieser Nachteil wird durch eine sehr einfache und damit schnelle Berechnung der Empfehlungen ausgeglichen. Zudem wurde gezeigt, dass die Fehlermaße nicht von denen andere Matrixfaktorisierungsmethoden abweichen. \citep{Cacheda2011}
 \newpage
