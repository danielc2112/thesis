\subsection{Recommendation Konzepte}\label{sec:concept}
Die Auswahl von möglichst relevanten Empfehlungen für einen Nutzer kann auf sehr verschiedenen Wegen getroffen werden. Für die zahlreichen bekannten Techniken wird in der Literatur vorwiegend die folgende Gliederung genutzt \citep[Kap. 1]{hb} \citep{Burke:2002:HRS:586321.586352} \citep{rs}:

\begin{itemize}
\item \textit{Kollaboratives Filtern}, auch \textit{\acf{CF}}, gewinnt relevante Elemente aus dem Vergleich des Nutzerprofils mit anderen  (ähnlichen) Profilen.
\item \textit{Gruppen-basierte Empfehlungen}, bzw. \textit{Community-based Filtering}, nutzen die Ähnlichkeit innerhalb von Gruppen, etwa in sozialen Netzwerken, um relevante Elemente zu finden.
\item \textit{Demographisch gestützte Empfehlungen} leiten sich von den Stereotypen, denen ein Nutzer zugeordnet wird, ab.
\item \textit{Inhaltsbasierte Empfehlungen} oder \textit{Content-based Recommendations}, werden auf der Basis von, am Nutzerprofil gewichteten, Element-Eigenschaften getroffen.
\item \textit{Wissensbasierte Empfehlungen} bzw. \textit{Knowledge-based Recommendations} werden durch zusätzliches domänenspezifisches Wissen generiert.
\item \textit{Utility-basierte Empfehlungen} bestimmen sich durch die Berechnung der ``Nützlichkeit'' der Elemente für den Nutzer mit Hilfe der sog. \textit{Utility Function}.
\item \textit{Hybride Systeme} kombinieren verschiedene Techniken um die Schwächen der einzelnen auszugleichen.
\end{itemize}

Die diesen Gruppen zugrunde liegenden Methoden werden in den nächsten Abschnitten näher erläutert. Dazu werden jeweils die zu erhebenden Daten, deren Verarbeitung und die Vor- und Nachteile der Methode beschrieben. % sowie mögliche Anwendungsgebiete beschrieben. 

%%%%%%%%%%%%%%%%%%%%%%%%%%%%%%%%%%%%%%%%%%%%%%%%%%%%%%%%%%%%%%%%%%%%%%%%%%%%%%
\subsubsection{Kollaboratives Filtern}
\label{sec:cf_overview}
Der Grundgedanke beim kollaborativen Filtern ist, dass Nutzer die in der Vergangenheit gleiche Interessen hatten, diese auch in der Zukunft durch ähnliches Verhalten ausdrücken. So können Empfehlungen für einen Nutzer aus dem Verhalten ähnlicher Nutzer abgeleitet werden. Die Nutzerprofile bilden sich dabei ausschließlich aus Elementbewertungen (\textit{Ratings}), Eigenschaften der bewerteten Elemente fließen nicht ein.  Die Ähnlichkeit der Nutzer drückt sich entsprechend durch Gemeinsamkeiten in den Bewertungen aus. \citep[Kap. 2]{rs}

Aus den Profilen aller Nutzer ergibt sich eine sog. \textit{User-Item} Matrix, diese ermöglicht es ähnliche Nutzer oder auch ähnliche Elemente im System zu finden. Zur Auswertung dieser Matrix, bzw. zur Generierung von Empfehlungen mit Hilfe dieser Matrix existieren verschiedene Strategien welche in Abschnitt \ref{sec:filtermethods} näher beschrieben werden.

Die Erhebung der Ratings kann sowohl auf explizite Weise, etwa mit einer 5-Punkte-Likert-Skala, oder implizit, zum Beispiel durch die Aufzeichnung von Browsing-Verläufen, geschehen.

Ein wichtiger Vorteil des kollaborativen Filterns liegt darin, dass Empfehlungen unabhängig von Elementeigenschaften gebildet werden können. Dadurch ist es möglich auch Elemente, deren Inhalt nur schwer oder gar nicht gewonnen werden kann, in die Empfehlung einzubeziehen. Die zahlreichen Forschungsarbeiten und die große Zahl der daraus hervorgegangenen Filterstrategien ist ebenfalls ein Vorteil.

Problematisch ist die Verwendung bei Systemen in denen der Nutzer (noch) kein oder nur ein sehr begrenztes Profil hat (\textit{Cold Start}). Zudem ist es nicht in jedem Fall sinnvoll alle Eigenschaften der Elemente ausser Acht zu lassen, da so ggf. problemspezifische Entscheidungskriterien unbeachtet bleiben.  \citep{hb,Burke:2002:HRS:586321.586352} %zum Beispiel der gewünschte Einsatzzweck beim Kauf einer Digitalkamera.%

\subsubsection{Gruppen-basierte Empfehlungen}\label{sec:cbf_overview}
Gemäß \citep{SinhaS01} haben Nutzer ein größeres Vertrauen in Empfehlungen wenn sie von Freunden ausgesprochen werden. Diesem Ansatz folgend werden in community-basierten Systemen Empfehlungen entsprechend der Präferenzen der Freunde eines Nutzers ausgesprochen. Das Nutzerprofil bildet sich daher aus einer Liste von Elementbewertungen und einer Liste von sozialen Verbindungen zu anderen Nutzern.

Da in Vergleichen mit reinen kollaborativen Systemen keine eindeutige Verbesserung der Empfehlungen nachgewiesen werden konnte, stellt das größere Vertrauen in die gebotenen Empfehlungen den wesentlichen Vorteil dieser Methode dar. Die gute Verbreitung und Verfügbarkeit der Daten über öffentliche Schnittstellen von  bestehenden sozialen Netzwerken, wie etwa Facebook oder LinkedIn, sind ebenfalls positiv. Die Abwägung zwischen der Aufrechterhaltung der Privatsphäre und dem dadurch resultierenden Verlust an Genauigkeit ist ein wichtiges Problem (vgl. \citep{machanavajjhala:accurate}). Auch das fehlende theoretische Fundament in anderen Bereichen, etwa beim Aufbau von Vertrauen und Misstrauen zwischen Nutzern, birgt mögliche Probleme bei der Umsetzung. \citep{hb_20}

Ein etwas anderer Gruppen-basierte Ansatz wird in \citep{smyth05a} beschrieben. Nimmt man an, dass eine bestimmte Plattform, wie zum Beispiel eine themenspezifsche Suchmaschine, nur von einer bestimmten Nutzergruppe (zB. Forscher) genutzt wird, sind die kollaborativ gewonnenen Empfehlungen ebenfalls gewinnbringend für die Gruppenmitglieder.

%%%%%%%%%%%%%%%%%%%%%%%%%%%%%%%%%%%%%%%%%%%%%%%%%%%%%%%%%%%%%%%%%%%%%%%%%%%%%%
\subsubsection{Demographisch gestützte Empfehlungen}
Eine weitere Methode um ähnliche Nutzer zu finden ist die Gruppierung nach demographischen Eigenschaften. So können Gruppen zum Beispiel entsprechend des Alters, der Sprache oder des Geschlechts gebildet werden. Sie können allerdings auch mit Hilfe der Methoden des maschinellen Lernens aus bestehenden Transaktionsdaten gewonnen werden (vgl. \citep{Burke:2002:HRS:586321.586352}). Wie bei den vorangegangenen Methoden bildet sich auch hier das Nutzerprofil zunächst aus einer Liste von Elementbewertungen, ergänzt wird es durch die entsprechenden demographischen Eigenschaften. Die Empfehlungen für den einzelnen Nutzer ergeben sich aus seinen eigenen Präferenzen die endsprechend der Gruppenzugehörigkeit gewichtet werden.

Arbeiten zu reinen demographischen Systemen gibt es kaum. In vielen Fällen, wie etwa \citep{Vozalis:2007:USD:1243505.1243639} werden kollaborative Ansätze ergänzt  um eine Verbesserung der Empfehlungsergebnisse zu erzielen bzw. um die Probleme bei Empfehlungen für neue Nutzer zu verringern. \citep{Burke:2002:HRS:586321.586352}
% http://dx.doi.org/10.1016/j.ins.2007.02.036

%%%%%%%%%%%%%%%%%%%%%%%%%%%%%%%%%%%%%%%%%%%%%%%%%%%%%%%%%%%%%%%%%%%%%%%%%%%%%%
\subsubsection{Inhaltsbasierte Empfehlungen}
Bei der inhaltsbasierten Generierung von Empfehlungen werden die Element-Ratings eines Nutzers zur Erzeugung eines ``Interessenprofils'' genutzt. In diesem Profil drücken sich die Präferenzen des Nutzers für die inhaltlichen Eigenschaften der Elemente aus und so kann es direkt genutzt werdens um Ihm Elemente mit ähnlichen Eigenschaften zu empfehlen. Hat ein Nutzer also zum Beispiel ein ``Harry Potter'' Buch positiv bewertet, so kann man leicht schlussfolgern, dass auch andere Fantasy-Bücher empfohlen werden könnten.

Neben der automatischen Erstellung des Profils ist es auch möglich dieses explizit vom Nutzer zu erfragen. Abhängig vom Problemfeld kann dies schneller zu guten Empfehlungen führen und zur Steigerung des Vertrauens in die erzeugten Empfehlungen beitragen, vgl. \citep{hb_20}.

Zur Bestimmung ähnlicher Dokumente, bzw. zur Extraktion der relevanten Eigenschaften (\textit{Features}) werden abhängig vom Elementtyp verschiedene Methoden genutzt. Diese reichen von Entscheidungsbäumen über neuronale Netze bis hin zu Vektorraum-Verfahren (vgl. Abschnitt \ref{sec:filtermethods} und \citep[Kap. 3]{rs}). Die große Anzahl der dafür zur Verfügung stehenden Verfahren, die damit verbundenen Erfahrungen und das daraus abgeleitete Problembewusstsein ist einer der Vorteile. Wichtiger noch ist die Tatsache, dass inhaltsbasierte Empfehlungen unabhängig von der Größe des Systems bzw. von der Anzahl der Nutzer generiert werden können. Ein weiterer Vorteil ist, dass für die so gewonnenen Empfehlungen auch leichter Erklärungen für den Nutzer generiert werden können, was wiederum ein wichtiger Faktor zur Steigerung des Vertrauens in die Qualität ist.

Schwierigkeiten bei der Erzeugung von Empfehlungen ergeben sich wenn die für den Nutzer relevanten Eigenschaften nicht direkt ``messbar'' vorliegen. Zum Beispiel des Ästhetik eines Produktes oder die Nutzbarkeit einer Webseite lassen sich nur sehr schwer erfassen, können aber beim Vergleich zweier Elemente wichtiger sein als textuelle Eigenschaften. Wie auch beim kollaborativen Filtern ist es bei dieser Methode sehr schwer gute Empfehlungen für Nutzer zu generieren, wenn diese kein oder nur ein unvollständiges Profil haben. Eine weitere Schwierigkeit ergibt sich daraus, dass Empfehlungen nur aus dem ``bevorzugten'' Interessenbereich des Nutzers gewonnen werden, dies kann zu sehr ähnlichen und kaum ``überraschenden'' Empfehlungen führen und zu einem Problem was als \textit{more of the same} umschrieben wird (vgl. Abschnitt \ref{sec:filterissues}).  \citep[Kap. 3]{rs} \citep{hb_03}

%Alternativ zur Ableitung des ``Interessenprofiles'' aus den Element-Ratings kann dieses auch direkt vom Nutzer erfragt werden. 

%%%%%%%%%%%%%%%%%%%%%%%%%%%%%%%%%%%%%%%%%%%%%%%%%%%%%%%%%%%%%%%%%%%%%%%%%%%%%%
\subsubsection{Wissensbasierte Empfehlungen}

Wenn die Frequenz mit der Nutzer ein Element brauchen oder konsumieren sehr gering ist, wie es etwa bei Hauskäufen der Fall ist, ergibt sich für die bisher beschriebenen Methoden das Problem, dass nur selten umfangreiche Nutzerprofile zur Verfügung stehen oder die darin enthaltenen Informationen schlicht veraltet sind. Oft gibt es zudem in vielen Bereichen Expertenwissen bzw. domänenspezifisches Wissen welches zur Verbesserung von Empfehlungen bzw. zur Einschränkung der Kandidatenliste genutzt werden kann.

Um dieses vorhandene Wissen zur Generierung von Empfehlungen nutzbar zu machen, kann man es in eine Menge von Regeln überführen und mögliche Empfehlungen entsprechend der Regeln filtern. So kann man zum Beispiel aus der Information, dass der Nutzer auf der Suche nach einer Wohnung für seine fünfköpfige Familie ist, leicht ableiten, dass 40$m^{2}$ Wohnungen nicht empfehlenswert sind und das solche mit zwei Bädern oder in einer ruhigeren Wohnlage empfohlen werden können.

Form und Inhalt des Nutzerprofils variieren hierbei in Abhängigkeit von der gewählten Wissens- bzw. Regelrepräsentation. Die Einbeziehung von Expertenwissen ermöglicht es auch übliche Standards einzubeziehen und es erleichtert die Vervollständigung des Nutzerprofils durch die Auswahl sinnvoller Fragen bei der Interaktion mit dem Nutzer. Auch in Fällen, in denen keine Vorschläge gefunden werden konnten, haben regelbasierte Systeme Vorteile. Die Information darüber das keine Empfehlungen für eine Anfrage gefunden werden konnten, werden Nutzer schneller akzeptieren wenn das System zudem eine Reihe von Vorschlägen unterbreiten kann welche der Regeln ausgelassen werden könnten um neue Empfehlungen zu generieren. Nachteile ergeben sich, wenn das Expertenwissen und die darauf basierenden Regeln nicht an neue Entwicklungen angepasst werden oder wenn für den Nutzer wichtige Features umbewertet bleiben. \citep[Kap. 4]{rs}

%%%%%%%%%%%%%%%%%%%%%%%%%%%%%%%%%%%%%%%%%%%%%%%%%%%%%%%%%%%%%%%%%%%%%%%%%%%%%%
\subsubsection{Utility-basierte Empfehlungen}

Ein zweiter Ansatz um domänenspezifisches Wissen zum Ausgangspunkt von Empfehlungen zu machen ergibt sich, indem man die ``Nützlichkeit'' eines Elements mit Hilfe einer nutzerspezifischen Funktion (\textit{Utility function}) berechnet. Dadurch kann zum Beispiel eine mögliche Toleranz des Nutzers gegenüber gewissen Produktmerkmalen direkt ins Verhältnis zur Dringlichkeit einer Bestellung gesetzt werden. Das Nutzerprofil ergibt sich dabei aus den Parametern der Funktion, welche i.d.R. explizit von Nutzer erfragt werden müssen.

Vor- und Nachteile sind ähnlich gelagert wie im vorangegangene Abschnitt. Vor allem der direkte Einfluss, den der Nutzer auf die Qualität der Ergebnisse hat, kann zur Steigerung das Vertrauens in die generierten Empfehlungen führen.  \citep[Kap. 1]{hb} \citep{Burke:2002:HRS:586321.586352, hb_20}


% Hybrid

% \citep{bogersVDBosch}


